\documentclass[oneside,UTF8]{article}
% ----------------------------------------
% ----------------------------------------
% ----------------------------------------
\usepackage{geometry}
\usepackage{fancyhdr}
\usepackage[heading=true]{ctex}
\usepackage{multicol}
\usepackage{datetime}
\usepackage[fontsize=12pt]{fontsize}
\usepackage{xcolor}
\usepackage{hyperref}
\usepackage{amsmath,amssymb,mathrsfs}
\usepackage{caption}
\usepackage{bm}
\usepackage{float}
\usepackage{graphicx}
\usepackage{listings}
\usepackage{lettrine}
% ----------------------------------------
% ----------------------------------------
% ----------------------------------------
\linespread{1.5}
\geometry{
    a4paper,
    left = 3.18cm,
    right = 3.18cm,
    top = 2.54cm,
    bottom = 2.54cm,
}
% ----------------------------------------
\hypersetup{
	colorlinks=true,
	% linkcolor=cyan,
	linkcolor=black,
	filecolor=blue,      
	% urlcolor=red,
	citecolor=cyan,
}
% ----------------------------------------
\lstset{
    extendedchars=false,
	columns = flexible,
	breaklines,
	showstringspaces = false,
	basicstyle = \normalsize,
	numbers = left,
	numberstyle = \tiny, 
	keywordstyle = \color{ blue!70},
	commentstyle = \color{red!50!green!50!blue!50}, 
	frame = shadowbox, 
	rulesepcolor = \color{ red!20!green!20!blue!20} ,
	xleftmargin=2em,xrightmargin=2em, aboveskip=1em,belowskip=1em,
	framexleftmargin=2em,framexrightmargin=2em,
	language = Python
}
% ----------------------------------------
\captionsetup[table]{labelfont={bf},labelformat={default},font={bf,normalsize},labelsep=space,name={Table}}
\captionsetup[figure]{labelfont={bf},labelformat={default},font={bf,normalsize},labelsep=space,name={Figure}}
% ----------------------------------------
\numberwithin{figure}{section}
\numberwithin{table}{section}
\numberwithin{equation}{section}
% ----------------------------------------
\CTEXsetup[name={,.},number={\Roman{section}}]{section}
\CTEXsetup[format={\Large\bfseries}]{section}
% ----------------------------------------
\newcommand*{\rightup}{\textsuperscript}
\newcommand*{\ucite}[1]{\textsuperscript{\cite{#1}}}
\renewcommand{\headrule}{\hbox to \headwidth{\color{red!20!green!20!blue!20} \leaders \hrule height \headrulewidth \hfill}}
% ----------------------------------------
\newcommand*{\purple}{\color{red!200!green!20!blue!20}}
\newcommand*{\gray}{\color[rgb]{0.5 0.5 0.5}}
\newcommand*{\red}{\color[rgb]{1 0 0}}
\newcommand*{\shallowRed}{\color[rgb]{1 0.2 0.5}}
\newcommand*{\shallowBlue}{\color[rgb]{0 0.98 0.93}}
\newcommand*{\shallowYellow}{\color[rgb]{1 0.9 0.6}}
\definecolor{shallowRed}{rgb}{1 0.2 0.5}
\definecolor{shallowGreen}{rgb}{0.88 0.93 0.85}
\definecolor{shallowYellow}{rgb}{1 0.9 0.6}
\definecolor{shallowBlue}{rgb}{0 0.98 0.93}
% ----------------------------------------
\title{\bf Clustering of Data from Four Regions}
\author{Illusionna}
\date{\tt\small \underline{18:05, Wednesday $\tt2^{nd}$ August, 2023} $\dashrightarrow$ \xxivtime,\ \today}
% ----------------------------------------
% ----------------------------------------
% ----------------------------------------
\begin{document}
% ----------------------------------------
\maketitle

\setcounter{page}{1}
\pagenumbering{arabic}
\fancyhead[L]{\purple\rightmark}
\fancyhead[R]{}
\fancyhead[C]{}
\fancyfoot[C]{\normalsize--\ \thepage\ --}
\thispagestyle{fancy}

\centerline{\bf\large 概述}

\begin{multicols}{2}
\lettrine[lines=2]{\bf 聚}{} 类形式丰富,可以分为有监督的聚类和无监督的聚类,譬如 K-means、GMM、SOM 等. 由于本人电脑性能不足,无法跑程序,学校服务器远程用不来,所以虚拟了一组数据 testData.xlsx 示例,以简单说明最优聚类数目的判断. 其次,由于聚类初始点的随机性,所以很多时候再次执行程序得到的最终聚类标签极大可能不尽相同,甚至所属类别都不同,这是正常的现象.

择优选取聚类数,仅仅我个人而言,之所以采用 TOPSIS 优劣解距离法思想,是因为考虑到 DBI 指数越小簇内性能越好而 DI 指数越大簇外(间)结果越好,而 TOPSIS 应该可以解决这样的问题. {\shallowRed 最后强调一下,这仅是一个示例,如果你赞同这种思想,就往下看.}

\end{multicols}
% ----------------------------------------
% ----------------------------------------
% ----------------------------------------
\pagestyle{fancy}
% ----------------------------------------
% ----------------------------------------
% ----------------------------------------
\section{准备数据执行程序}
查看 $\rm\bf{./Cluster/Programs/README.md}$ 准备好数据(100M 有点大所以我没有打包).



\section{一处解释}
三种聚类方法没有采取训练集和测试集的划分. 如果,我将 testData.xlsx 按照三七开,设置聚类数(假如 $\rm n\_clusters=4$),去拿训练集训练得到的模型预测测试集,那么,我会得到预测标签结果,就像:
\[ \mbox{测试集:}[2,\ 1,\ 0,\ 0,\ 3] \]

这表明,测试集第一行数据(第一个样本)属于第 2 类,第二个样本属于第 1 类,第三个样本属于第 0 类,以此类推. 起源数据的标签只有 0、1 两种(人为设置的标签,可以看作有监督的),而我们测试集预测的数据却反映 0、1、2、3 四类,显而易见,牛头不对马嘴.

而且,即便设置聚类数 $\rm n\_clusters=2$,也存在这样一个现象.
\[ \mbox{第一次执行测试集:}[0,\ 1,\ 0,\ 0,\ 1] \]
\[ \mbox{第二次执行测试集:}[1,\ 0,\ 1,\ 0,\ 1] \]
\[ \mbox{第二次执行测试集:}[1,\ 1,\ 0,\ 0,\ 0] \]

在起源数据中,假设我们监督的测试集第一个样本标签是 0,但多次执行程序,未必见得预测的第一个样本就一定隶属第 0 类,这个例子中,第一个样本在三次执行情况下分别隶属第 0、1、1 类.

假设我们人类认为的标签 0 代表“迦南”,1 代表“安可”,那就是说,测试集第一个样本被标记为迦南,但现在预测结果反映第一个样本第一次被机器认为是迦南,但第二次第三次被认为是安可.

迦南被认为机器判为迦南$\to$安可$\to$安可,若我们采用交叉熵评判:
\[ \rm accuracy=\frac{1}{3} \]

{\bf\red 但},很可能我下次执行得到:
\[ \mbox{第一次执行测试集:}[0,\ 1,\ 0,\ 0,\ 1] \]
\[ \mbox{第二次执行测试集:}[0,\ 0,\ 1,\ 0,\ 1] \]
\[ \mbox{第二次执行测试集:}[1,\ 1,\ 0,\ 0,\ 0] \]

准确率:
\[ \rm accuracy=\frac{2}{3} \]

更重要的是,这只是测试集一个样本,而测试集有起源数据$\times30\%$ 的量,这会使得多次执行程序得到的交叉熵判断混乱,第一次可能是迦南判为迦南,安可判为安可,到了第二次就变成迦南判为安可,安可判为安可,第一次的 accuracy 和第二次的 accuracy 可能天差地别,交叉熵不稳定.

正是鉴于上面这两种现象,所以没有将起源数据划分训练集测试集,如果,使用者有划分需求,则需要自行补充相关 Python 函数.



\section{聚类结果解读}
自动生成相应聚类算法 Results 子文件夹.
\begin{figure}[H]
	\centering
	\includegraphics*[scale=0.3]{./images/KmeansResultsFileFolder.png}
	\caption{./Cluster/Programs/Kmeans}
\end{figure}

testData.xlsx 聚类数据放置在 Results 文件夹下.
\begin{figure}[H]
	\centering
	\includegraphics*[scale=0.3]{./images/testDataFileFloder.png}
	\caption{./Cluster/Programs/Kmeans/Results}
\end{figure}

左边文件夹存放聚类指数 DBI 和 DI 结果,后续可用于 TOPSIS,右边文件夹存放聚类结果,看需要使用.
\begin{figure}[H]
	\centering
	\includegraphics*[scale=0.3]{./images/testDataSubFileFoldaer.png}
	\caption{Iteration Results}
\end{figure}



\section{综合评价}
这里提供之前 Matlab 函数 $\rm\bf{./Cluster/Programs/TOPSIS.m}$,可能需要使用者自行修改读取 .txt 文件数据,应该包括“路径”和“读取跳跃的步长”.

\lstset{language=Matlab}
\begin{lstlisting}
M = importdata("./SOM_Intrinsic_Exponential/Intrinsic_Exponential.txt").data;
N = length(M);
GMM_DBI = M(1:2:N-1);
GMM_DI = M(2:2:N);
\end{lstlisting}

最后,待使用者跑完数据,仿照下面形式择优选取较佳的聚类数目.

\begin{figure}[H]
	\centering
	\includegraphics*[scale=0.18]{./images/demo.pdf}
\end{figure}




\end{document}